\documentclass[12pt]{article} % Uses 10pt

\usepackage{fancyhdr}
\usepackage{helvet}
\usepackage{url}
\textwidth6.5in
\textheight=9.0in  % text height can be bigger for a longer letter
\setlength{\topmargin}{0.3in} 
\addtolength{\topmargin}{-\headheight}
\addtolength{\topmargin}{-\headsep}
\headsep = 20pt


\renewcommand{\baselinestretch}{1.1} 

\setlength{\oddsidemargin}{0in}

\oddsidemargin  0.0in \evensidemargin 0.0in 

%%\parindent0em

\pagestyle{fancy}
\renewcommand{\headrulewidth}{0pt}
\lhead{}
\rhead{}
\chead{\bf \large Stat 518 Project Report Guidelines} 
\lfoot{} 
\rfoot{} 
\cfoot{}


\newenvironment{enumerate*}%
  {\begin{enumerate}%s
    \setlength{\itemsep}{0.01cm}%
    \setlength{\parskip}{0.05cm}%
    \setlength{\abovedisplayskip}{0.00cm}%
    \setlength{\belowdisplayskip}{0.00cm}}%
  {\end{enumerate}}

\newenvironment{itemize*}%
  {\begin{itemize}%
    \setlength{\itemsep}{0.01cm}%
    \setlength{\parskip}{0.02cm}%
    \setlength{\abovedisplayskip}{-0.12cm}%
    \setlength{\belowdisplayskip}{-0.12cm}}%
  {\end{itemize}}

\newcommand{\sectionname}[1]{\vspace{0.23cm} \noindent {\bf #1}}



\begin{document}

\par
%%\noindent
The written report should not exceed 15 pages, including all supporting materials. You should 
submit your \underline{documented source code} as an appendix, 
which does not count toward the 15 page limit.
The actual number of pages in your report is irrelevant as long as this number does not exceed the 
limit. In other words, \underline{it is OK to submit a 10 page report}. 
The report should contain a description of the statistical problem and explain
scientific motivation behind this problem. \textbf{It is very important that your report does not merely
summarize the paper that you have been reading this quarter}, but also includes discussion of relevant
literature, competing methods, applications of the methodology developed in the paper, etc. 
\par
A substantial portion of your report should be devoted to technical details. You should go beyond 
equations that are already in the paper. To accomplish this, you can provide alternative model 
formulations, make meaningful connections to other models/methods, and fill in gaps in derivations or 
less formal analytical arguments of the paper. This section of your proposal should clearly demonstrate
that you understand technical details of the paper.
\par
Next, you should provide evidence that you reproduced a significant part of the paper computational
results. Here, you will discuss your simulations, statistical estimation, model selection, and 
other relevant numerical results. This is probably the easiest section to fill in, so try to be 
succinct and concentrate only on key results.
\vspace{0.4cm}
\par
\noindent
You should keep in mind three main themes, around which your report should revolve:
\begin{itemize*}
\item Stochastic modeling
\item Statistical inference
\item Scientific problem
\end{itemize*}
The instructor will pay attention to all three themes during the evaluation of your report and 
presentation.
\end{document}






